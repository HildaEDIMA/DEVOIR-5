% Options for packages loaded elsewhere
\PassOptionsToPackage{unicode}{hyperref}
\PassOptionsToPackage{hyphens}{url}
%
\documentclass[
]{article}
\usepackage{amsmath,amssymb}
\usepackage{iftex}
\ifPDFTeX
  \usepackage[T1]{fontenc}
  \usepackage[utf8]{inputenc}
  \usepackage{textcomp} % provide euro and other symbols
\else % if luatex or xetex
  \usepackage{unicode-math} % this also loads fontspec
  \defaultfontfeatures{Scale=MatchLowercase}
  \defaultfontfeatures[\rmfamily]{Ligatures=TeX,Scale=1}
\fi
\usepackage{lmodern}
\ifPDFTeX\else
  % xetex/luatex font selection
\fi
% Use upquote if available, for straight quotes in verbatim environments
\IfFileExists{upquote.sty}{\usepackage{upquote}}{}
\IfFileExists{microtype.sty}{% use microtype if available
  \usepackage[]{microtype}
  \UseMicrotypeSet[protrusion]{basicmath} % disable protrusion for tt fonts
}{}
\makeatletter
\@ifundefined{KOMAClassName}{% if non-KOMA class
  \IfFileExists{parskip.sty}{%
    \usepackage{parskip}
  }{% else
    \setlength{\parindent}{0pt}
    \setlength{\parskip}{6pt plus 2pt minus 1pt}}
}{% if KOMA class
  \KOMAoptions{parskip=half}}
\makeatother
\usepackage{xcolor}
\usepackage[margin=1in]{geometry}
\usepackage{graphicx}
\makeatletter
\def\maxwidth{\ifdim\Gin@nat@width>\linewidth\linewidth\else\Gin@nat@width\fi}
\def\maxheight{\ifdim\Gin@nat@height>\textheight\textheight\else\Gin@nat@height\fi}
\makeatother
% Scale images if necessary, so that they will not overflow the page
% margins by default, and it is still possible to overwrite the defaults
% using explicit options in \includegraphics[width, height, ...]{}
\setkeys{Gin}{width=\maxwidth,height=\maxheight,keepaspectratio}
% Set default figure placement to htbp
\makeatletter
\def\fps@figure{htbp}
\makeatother
\setlength{\emergencystretch}{3em} % prevent overfull lines
\providecommand{\tightlist}{%
  \setlength{\itemsep}{0pt}\setlength{\parskip}{0pt}}
\setcounter{secnumdepth}{-\maxdimen} % remove section numbering
\usepackage{xcolor}
\usepackage{fontawesome}
\usepackage{graphicx}
\pagestyle{empty}
\ifLuaTeX
  \usepackage{selnolig}  % disable illegal ligatures
\fi
\IfFileExists{bookmark.sty}{\usepackage{bookmark}}{\usepackage{hyperref}}
\IfFileExists{xurl.sty}{\usepackage{xurl}}{} % add URL line breaks if available
\urlstyle{same}
\hypersetup{
  hidelinks,
  pdfcreator={LaTeX via pandoc}}

\author{}
\date{\vspace{-2.5em}}

\begin{document}

\definecolor{lavender}{HTML}{E6E6FA}

\begin{center}
\textcolor{lavender}{\LARGE\textbf{Votre Nom}} \\
\textcolor{lavender}{Votre Adresse} | \faMobilePhone\ +33 6 12 34 56 78 | \faEnvelopeO\ votre.email@example.com \\
\end{center}

\hrulefill

\hypertarget{lettre-de-motivation}{%
\section{Lettre de Motivation}\label{lettre-de-motivation}}

Le {[}Date{]}

{[}Prénom Nom du Directeur{]} \textbackslash{} Directeur
\textbackslash{} Institut National de la Statistique (INS)
\textbackslash{} {[}Adresse{]} \textbackslash{} {[}Ville, Code Postal{]}
\textbackslash{} {[}Cameroun{]}

\hrulefill

Cher {[}Prénom Nom du Directeur{]},

Je m'adresse à vous dans le cadre d'une demande de stage au sein de
l'Institut National de la Statistique (INS) du Cameroun. Actuellement
étudiant(e) en {[}Votre Domaine d'Études{]} à {[}Nom de votre
Établissement{]}, je suis passionné(e) par l'analyse des données et
l'application des méthodes statistiques dans la résolution de problèmes
concrets.

Au cours de mon parcours académique, j'ai acquis des compétences solides
en analyse statistique, en programmation et en manipulation de données.
Je suis familier(e) avec des outils tels que R, Python et SQL, et j'ai
déjà eu l'occasion de les mettre en pratique lors de projets académiques
et de stages précédents. Je suis convaincu(e) que mon expertise
technique et ma passion pour les statistiques pourraient être mises à
profit au sein de votre équipe à l'INS.

Je suis particulièrement intéressé(e) par {[}mentionnez un aspect
spécifique du travail de l'INS, par exemple, la collecte de données,
l'analyse démographique, etc.{]}. Je suis motivé(e) à contribuer à vos
projets et à apprendre de nouvelles techniques sous votre direction et
celle de votre équipe.

Je suis disponible pour un stage d'une durée de {[}durée du stage{]} à
partir du {[}date de début souhaitée{]}. Je serais honoré(e) de pouvoir
discuter de ma candidature lors d'un entretien. Je reste à votre
disposition pour toute information supplémentaire que vous pourriez
avoir besoin.

Je vous remercie de l'attention que vous porterez à ma candidature. Je
suis enthousiaste à l'idée de pouvoir contribuer à l'INS du Cameroun et
je suis convaincu(e) que ce stage serait une opportunité enrichissante
pour mon développement professionnel.

Je vous prie d'agréer, Cher {[}Prénom Nom du Directeur{]}, l'expression
de mes salutations distinguées.

\bigskip

Cordialement,

{[}Votre Nom{]}

\end{document}
